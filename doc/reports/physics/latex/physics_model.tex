\documentclass[a4paper,12pt]{article}
\usepackage[latin1]{inputenc} %entrada usando ISO-Latin1
\usepackage{listings}
\usepackage{color}
\usepackage{a4wide}
\usepackage{graphicx}
\usepackage{subfig}
\usepackage{wrapfig}
\usepackage{hyperref}
\usepackage{amsmath}
\usepackage{tikz}
\usetikzlibrary{shapes, arrows}
\usepackage{appendix}

\usepackage{bm, amssymb, pifont, pgfplots, multirow, float, soul, color, verbatim, indentfirst}

\newcommand{\be}{\begin{equation}}
\newcommand{\ee}{\end{equation}}

\lstset{
   breaklines=true,
   basicstyle=\ttfamily}
% \usepackage{draftwatermark}\usepackage[figurewithin=section,tablewithin=section]{caption}
%\usepackage{subfigure}
\setcounter{secnumdepth}{4}

\pagestyle{headings}

\title{\textbf{LCLS-II System simulations:\\Back-end}}
\author{LBNL LLRF Team}

\date{\today}

\begin{document}
\numberwithin{equation}{section} 
\maketitle
\setcounter{tocdepth}{3}
\tableofcontents

\newpage

\section{Overview}

\begin{figure}
\centering
\includegraphics[scale=0.11]{../figures/LCLS-II_feedback_layout.png}
\caption{LCLS-II feedback layout.}
\label{fig:lclsII-feedback_layout}
\end{figure}

During the design, comissioning and operations of a Linac-driven FEL it is useful to have modeling capabilities to abstract and analyze some of the complex problems involved in LLRF and beam-based feedback in the confortable environment of computer simulations. The simulation framework presented here is a dramatic improvement of a previous version written in Octave/Matlab~\cite{ref:model-paper}, where extensively tested LLRF models are integrated with a longitudinal phase space tracking simulator~\cite{ref:litrack} along with the interaction between the two via beam-based feedback using a computionally efficient simulation engine. 

The models include beam instrumentation, considerations on loop delays for in both the LLRF and beam-based feedback loops, as well as the ability to inject noise (both correlated and uncorrelated) at different points of the machine including a full characterization of the electron gun performance parameters. The Linac is divided into generic compounds composed of an accerating section followed by a bunch compressor (where the bunch compressor can be enabled/disabled) and beam performance parameters are measured at any stage of the machine for characterization and/or for use to apply beam-based feedback. Time-series data is computed at a configurable repetition rate and results can be visualized in both time and frequency domain, including transfer functions between any noise source and beam performace parameters.


Figure~\ref{fig:lclsII-feedback_layout} shows a high-level representation of the LCLS-II layout, as configured at the time of this writing. The model described here represents each component of the machine in a way that configuration parameters can be adapted as the machine layout and configuration evolves (during the design or in future upgrades), where the contribution of each noise source (correlated or uncorrelated) to the machine performance budget can be quantified, including the ability to represent the effectiveness of different feedback loops to reduce noise contributions under different configurations. 

\section{Model hierarchy}

\begin{figure}
\centering
\includegraphics[scale=0.6]{../figures/Model_hierarchy.png}
\caption{Model hierarchy.}
\label{fig:model_hierarchy}
\end{figure}

The accelerator model is intended to be modular in order to adapt to different configurations. The motivation behind this partitioning is to allow for different types of studies. For example, one can focus on some localized effects at the RF station level where only one station is simulated, or inspect how different cavities interact mechanically inside a cryomodule, or run studies at a machine level in order to analize slow beam-based feedback performance where not much detail on the internals of RF stations is desired.

The different model configurations described above are put in practice using a model hierarchy shown in Figure~\ref{fig:model_hierarchy}. The first component is a simulation entity, which describes simulation parameters such as time step size, total simulation time, etc. The simulation then includes an accelerator which can be composed of one or more Linac sections, including a longitudinal beam dynamics simulation to represent beam propagating from one Linac section to the next. One linac section is then represented as a series of cryomodules (or modules), optinally followed by a bunch compressor. This is the fundamental building block of the accelerator, where one can imagine to have the case of L1 in Fig.~\ref{fig:lclsII-feedback_layout} with two cryomodules and no bunch compressor, followed by HL (with different frequency, beam phase relative to the RF, etc), this time followed by a bunch compressor (BC1 in this case).

Each module is then represented by a collection of RF stations, including inter-station interactions through a model of the mechanical resonances in the cryomodule. The electro-mechanical couplings between the mechanical modes and individual electrical modes in the cavities are represented, as well as the effect of tuners and piezos on these resonances. Each station is composed of the typical RF system layout, with an N-cell cavity (including several normal modes), a high-power RF source, an FPGA controller, and the analog front-end (which represents anti-alias filtering, LLRF noise. etc). The last stage of the hierarchy is the cavity electrical modes, which each have their own resonance frequency, Q and couplings with mechanical modes in a module.

In the next few sections we describe the models used in each layer of the hierarchy shown in Fig.~\ref{fig:model_hierarchy}, starting bottom up so that the reader progressively understands every component in each layer without making any assumptions.

\newpage

\section{RF Station}

\begin{figure}
\centering
\includegraphics[scale=0.75]{../figures/Station_rf_blocks.png}
\caption{Station block diagram.}
\label{fig:Station_block_diagram}
\end{figure}

The model includes a multi-cell cavity (inlcuding different normal modes, detuning and their different couplings), an FPGA controller, a saturation model for the RF source, filters, etc. The RF station is modeled at baseband, where up and down conversions in the real system are not considered and only the slowly varying amplitude and phase modulations (or In-phase and Quadrature, I\&Q) of a carrier at the RF reference frequency are represented. The FPGA controller in a real system typically works on I-Q sampled cavity fields. Representing the rest of the components in the RF station model at baseband allows for a more computationally efficient implementation of the simlation code, while not losing information of interest on the different signals. 

The RF station model responds to the typical RF system topology, with an FPGA-based system controlling the EM field inside an accelerating cavity as shown in Fig.~\ref{fig:Station_block_diagram}. The different components of the RF station model are described in the rest of this section.

\subsection{Cavity model}

In each cavity of a particle accelerator, we want the highest possible $\vec{E}$ field amplitude along the particles' acceleration direction in order to transmit the maximum energy to them.  From EM theory, we know $\vec{E}$ and $\vec{B}$ fields inside a cavity can be broken down into independent eigenmodes (independent solutions to Maxwell's equations inside a cavity or waveguide). For a 9-cell cavity, the highest $\vec{E}$ field along the cavity axis is obtained for the $\pi$-mode. The electrical input to apply to a cavity to obtain that mode only is known from theory. However, it is hard to reach in practice as some other modes are usually present as well due to geometrical errors in the cavity shape or further deformation due to mechanical forces that are generated.

The cavity model described here responds to the typical multi-cell cavity structure, with couplings to the RF source, a probe, and the beam of particles. There are two aspects to representing this problem: the complete electro-magnetic field description inside the cavity and the equivalent circuit representation, where the field description is needed in order to define the equivalent-circuit using simulation codes like Superfish~\cite{ref:superfish}. It is convinient to use the equivalent-circuit representation in order to model the cavity behavior as well as its interactions with the RF system and the beam.

Ideally, we would like to measure the EM fields from each mode present in the cavity in order to control them appropriately. However, the best  we can do is measure the overall field in the cavity, designated here by $\vec E_{\rm probe}$. It is measured in practice using a probe, and is theoretically given by

\begin{equation}
  \vec E_{\rm probe} = \sum_{\mu} \vec{V}_{\mu} /\sqrt{Q_{\rm p_{\mu}}(R/Q)_{\mu}}
  \label{eq:E_probe}
\end{equation}

\noindent where $\vec{V}_{\mu}$ is a representative measure of the energy stored in each electrical mode $\mu$, designated as mode cavity voltage, and where $Q_{\rm p_{\mu}}(R/Q)_{\mu}$ is the coupling impedance of the probe port for that mode.

Alternatively, the expression for reverse (a.k.a.~reflected) wave travelling outward from the fundamental port includes a prompt reflection term, yielding

\begin{equation}
  \vec E_{\rm reverse}=\sum_\mu \vec V_\mu / \sqrt{Q_{g_{\mu}}(R/Q)_\mu} - \vec K_g
  \label{eq:E_reverse}
\end{equation}

\noindent where $Q_{g_{\mu}}(R/Q)_\mu$ is the coupling impedance of the drive port of mode $\mu$.

\subsubsection{Electromagnetic eigenmode}

\begin{figure}
\centering
\includegraphics[scale=0.20]{../figures/cavity_eq_circuit.png}
\caption{Electromagnetic eigenmode equivalent circuit.}
\label{fig:cav_eq_circuit}
\end{figure}

A multi-cell cavity is represented by a series of coupled resonators (one per cell in the cavity), each respresented by an RLC circuit~\cite{ref:montgomery}. The equivalent circuit used to represent one cavity mode is shown in Fig.~\ref{fig:cav_eq_circuit}, where each mode's accelerating voltage is added in order to obtain the cavity overall accelerating voltage~\cite{ref:cell_modes}, as deduced from Equation~\ref{eq:E_probe}. Each mode has its own value of $\vec V$, $(R/Q)$, $Q_x$, and other characteristics that will be introduced later.

If we apply Kirchhoff's current law to the mode's RLC equivalent circuit (see figure~\ref{fig:cav_eq_circuit}), we get:

\begin{equation}
  \centering \vec I_{\mu} = \vec I_{\rm C_{\mu}} + \vec I_{\rm R_{\mu}} + \vec I_{\rm L_{\mu}}
  \label{eq:currents}
\end{equation}

\noindent where:

\begin{equation}
  \frac{d\vec I_{\rm C_{\mu}}}{dt} = C_{\mu} \cdot \frac{d^2\vec V_{\mu}}{dt^2}\text{,}\qquad \frac{d\vec I_{\rm R_{\mu}}}{dt} = \frac{1}{R_{\rm L_{\mu}}}\frac{d\vec V_{\mu}}{dt} \qquad \text{and}\qquad \frac{d\vec I_{\rm L_{\mu}}}{dt} = \vec V_{\mu}/L_{\mu}
  \label{eq:currents2}
\end{equation}

Differentiating both sides of equation~\ref{eq:currents} and substituting using eq.~\ref{eq:currents2}, the full vector (complex) differential equation for the cavity accelerating voltage $\vec V_{\mu}$ can be written as:

\begin{equation}
  \frac{d^2\vec V_{\mu}}{dt^2} + \frac{1}{R_{\rm L_{\mu}}C_{\mu}}\frac{d\vec V_{\mu}}{dt} + \frac{1}{L_{\mu}C_{\mu}}\vec V_{\mu} = \frac{1}{C_{\mu}}\frac{d\vec I_{\mu}}{dt}
\end{equation}

\noindent which can be expressed as a function of the mode's nominal resonance frequency $\omega_{0_{\mu}}$ ($1/L_{\mu}C_{\mu}=\omega_{0_{\mu}}^2$) and loaded Q ($1/R_{\rm L_{\mu}}C_{\mu}=\omega_{0_{\mu}}/Q_{\rm L_{\mu}}$):
 
\begin{equation}
  \frac{d^2\vec V_{\mu}}{dt^2} + \frac{\omega_{0_{\mu}}}{Q_{\rm L_{\mu}}}\frac{d\vec V_{\mu}}{dt} + \omega_{0_{\mu}}^2 \vec V_{\mu} = \frac{\omega_{0_{\mu}}^2 R_{\rm L_{\mu}}}{Q_{\rm L_{\mu}}}\frac{d\vec I_{\mu}}{dt}
  \label{eq:2nd_order}
\end{equation}

Taking the slowly varying envelope approximation~\cite{ref:svea} and separating voltage and current into real and imaginary parts we can reduce the order of equation~\ref{eq:2nd_order} (a second-order band-pass filtered centered at the resonance frequency) to a first-order low-pass filter at baseband~\cite{ref:schilcher}:

\begin{equation}
  \left(1-j\frac{\omega_{d_{\mu}}}{\omega_{f_{\mu}}}\right)\vec V_{\mu} + \frac{1}{\omega_{f_{\mu}}}\frac{d\vec V_{\mu}}{dt} = R_{\rm L_{\mu}} \vec I_{\mu}
  \label{eq:1st_order1}
\end{equation}

\noindent where $\omega_{f_{\mu}}=\omega_{0_{\mu}}/2Q_{L_{\mu}}$ is the mode's bandwidth and $\omega_{d_{\mu}}=2\pi\Delta f_{\mu}$ is the (time varying) detune frequency,  , given as $\omega_{d_{\mu}}=\omega_{0_{\mu}}-\omega_{ref}$, {\it i.e.}, the difference between actual eigenmode frequency $\omega_{0_{\mu}}$ and the accelerator's time base $\omega_{ref}$.

Transposing the cavity drive term into a combination of the RF source incident wave and beam loading (opposite sign indicating energy absortion by the beam), we can express eq.~\ref{eq:1st_order1} as:

\begin{equation}
  \left(1-j\frac{\omega_{d_{\mu}}}{\omega_{f_{\mu}}}\right)\vec V_{\mu} + {\frac{1}{\omega_{f_{\mu}}}}\frac{d\vec V_{\mu}}{dt} =  2\vec K_{\rm g}\sqrt{R_{\rm g_{\mu}}} - R_{\rm b_{\mu}}\vec I_{\rm beam}
  \label{eq:1st_order2}
\end{equation}

\noindent where $\vec K_{\rm g}$ is the incident wave amplitude in $\sqrt{\rm Watts}$, $R_{\rm g_{\mu}}=Q_{\rm g_{\mu}}(R/Q)_{\mu}$ is the coupling impedance of the drive port, $\vec I_{\rm beam}$ is the beam current, and $R_{\rm b_{\mu}}=Q_{\rm L_{\mu}}(R/Q)_{\mu}$ is the coupling impedance to the beam.

The overall $Q_{\rm L_{\mu}}$ is given as $1/Q_{\rm L_{\mu}}=1/Q_{0_{\mu}}+1/Q_{g_{\mu}}+1/Q_{p_{\mu}}$, where $1/Q_{0_{\mu}}$ represents losses to the cavity walls, $1/Q_{g_{\mu}}$ represents coupling to the input coupler, and $1/Q_{p_{\mu}}$ represents coupling to the field probe. $(R/Q)_{\mu}$ is the shunt impedance of the mode in Ohms, a pure geometry term computable for each particular eigenmode using E\&M codes like Superfish. Physically, shunt impedance relates a mode's stored energy $U_{\mu}$ to the accelerating voltage it produces, according to 

\begin{equation}
  U_{\mu} = \frac{V_{\mu}^2}{(R/Q)_{\mu}\omega_{0_{\mu}}}
\end{equation}

The only assumptions in the above formulation are that the cavity losses are purely resistive, and thus expressible with a fixed $Q_{0_{\mu}}$, and that no power is launched into the cavity from the field probe.  If other ports have incoming power, there would be additional terms of the same form as $2\vec K_g\sqrt{R_g}$.

The $\frac{\omega_{d_\mu}}{\omega_{f}}\vec{V}_{\mu}$ term in \ref{eq:1st_order2} will be relatively small compared to the other terms. To appropriately take it into account it is better to isolate it by defining  a vector $\vec{S}_{\mu}$ such that $\vec{V}_{\mu} = \vec{S}_{\mu}e^{j\theta_{\mu}}$, with $d\theta_{\mu}/dt = \omega_{d_{\mu}}$, yielding: 

\begin{equation}
\left(1 - j\frac{\omega_{d_\mu}}{\omega_{f_{\mu}}}\right)\vec{S}_{\mu}e^{j\theta_{\mu}} + \frac{1}{\omega_{f_{\mu}}}
    \left(\frac{d\vec{S}_{\mu}}{dt}e^{j\theta_{\mu}} + \vec{S}_{\mu} \cdot j\omega_{d_{\mu}}e^{j\theta_{\mu}}\right) = 2\vec{K}_{g}\sqrt{R_{g_{\mu}}} - R_{b_{\mu}}\vec{I}_{\rm beam}
\label{eq Gov_S}
\end{equation}

\noindent The governing equation for the mode's accelerating voltage can thus be written as a set of two first order differential equations

\begin{equation}
\frac{d\theta_{\mu}}{dt} = \omega_{d_\mu}
\label{eq: d0_dt}
\end{equation}

\begin{equation}
  \frac{d\vec{S}_{\mu}}{dt} = -\omega_{f_{\mu}}\vec{S}_{\mu} + \omega_{f_{\mu}}e^{-j\theta_{\mu}}\left(2\vec{K}_g\sqrt{R_{g_{\mu}}} 
    - R_{b_{\mu}}\vec{I}_{\rm beam} \right)
\label{eq: dS_dt}
\end{equation}

Note that this state-variable equation is a pure low-pass filter, an advantage especially in the FPGA implementation.

\subsubsection{Software implementation}

\begin{figure}
\centering
\includegraphics[scale=0.4]{../figures/Cavity_modes.png}
\caption{Data path for the computation of a 9-cell cavity model.}
\label{fig:RF_cavity_block_diagram}
\end{figure}

We started this section with the definition of $\vec E_{\rm probe}$ and $\vec E_{\rm reverse}$ (see Fig.~\ref{fig:Station_block_diagram}), given by equations \ref{eq:E_probe} and \ref{eq:E_reverse} respectively. These equations express the measured probe and reverse fields as a function of eigenmode voltages ($\vec V_\mu$) and their respective port couplings. We also defined the state-variable equation governing the accelerating voltage for each eigenmode in Equation~\ref{eq: dS_dt}, where $\vec{V}_{\mu} = \vec{S}_{\mu}e^{j\theta_{\mu}}$. We can therefore compute the cavity probe and reverse signals as a function of the incident wave $\vec K_{\rm g}$ and the beam current $\vec I_{\rm beam}$.

Figure~\ref{fig:RF_cavity_block_diagram} shows the data path implemented in software in order to compute the response of a nince-cell cavity (which can be configured for any type of cavity given the definition of the electrical eigenmodes and couplings). Each internal bock represents the implementation of Eq.~\ref{eq: dS_dt} for each eigenmode, and the summing juntion at the end (taking the pre-factored eigenmode voltages) represents the computation of $\vec E_{\rm probe}$ and $\vec E_{\rm reverse}$ using Equations \ref{eq:E_probe} and \ref{eq:E_reverse}.

Note that there are two aspects represented in Fig.~\ref{fig:RF_cavity_block_diagram} which are not shown in the equations. The first one is the use of $\vec K_{\rm g_{rfl}}$ instead of $\vec K_{\rm g}$, as well as the terms $e^{j\varphi_{x}}$ in the factoring of each $\vec V_\mu$ term before the summing junctions. These terms represent frequency dependent propagation through cables and waveguides. In the case of the incident wave, if we define $TF_{\rm wg}(s)$ as the transfer function in Laplace domain of a wide-band filter representing the waveguide between the directional coumpler on the high-power forward path and and the cavity, then:

\begin{equation}
 \vec K_{\rm g_{rfl}}(s) = \vec K_{\rm g}(s) \cdot TF_{\rm wg}(s)
\end{equation}

\noindent where the minus sign of the reflection on the cavity coupler is represented in the summing junction. This transformation takes into account the propagation of the reflection of the incident wave back to the directional coupler.
The cavity probe and reverse path also follow a similar transformation but in this case represented by a phase shift through the coaxial cable from the cavity probe and reverse ports and their respective ADCs in the LLRF. These phase shifts are represented by the $e^{j\varphi_{\rm rev_{\mu}}}$ and $e^{j\varphi_{\rm p_{\mu}}}$ terms in Fig.~\ref{fig:RF_cavity_block_diagram}, which are frequency dependent due to dispersion in the coaxial cable and it is therefore convinient to apply at this stage of the computation (before the summing junction).

The second aspect which has not been covered yet is the numerical discretization of the first-order low-pass filter in the cavity response, represented by the blocks labeled $LPF@\omega_{f_{\mu}}$ in Fig.~\ref{fig:RF_cavity_block_diagram}. The ODE integration is derived in Appendix~\ref{App:ODE_integration}, where Equation~\ref{eq: dS_dt} can be written as Equation~\ref{eq:exp_diff2}, where

\begin{equation}
 \vec V_{\rm out} = \vec S_{\mu} \mbox{ , } p = -\omega_f \mbox{ and, } \vec V_{\rm in} = e^{-j\theta_{\mu}} \left(2\vec{K}_g\sqrt{R_{g_{\mu}}} - R_{b_{\mu}}\vec{I}_{\rm beam}\right)
\end{equation}

\noindent as it can be deduced from Fig.~\ref{fig:RF_cavity_block_diagram}. 

In summary, using the computations shown in Figure~\ref{fig:RF_cavity_block_diagram} (couplings, rotations, etc.) combined with the numerical discretization of the cavity filter described in Appendix~\ref{App:ODE_integration} applied every discrete simulation step, we can obtain time-series simulated data representing signal propagation through the multi-cell RF cavity, including the different eigenmodes and dispersion through cables and waveguides.

\subsubsection{Simulation results}


\subsection{FPGA Controller}

A Proportional-Integrator (PI) Controller is generally used in RF cavities for particle accelerators to control the input driving signal in order to generate the desired field. A flow chart of this controller is shown below, followed by the mathematical expressions. 

%Define block styles
\tikzstyle{block} = [rectangle, draw, node distance = 2.5cm, text width =16em, text centered, rounded corners, minimum height=2em]
\tikzstyle{line} = [draw, -latex']
\tikzstyle{cloud} = [draw, ellipse,  node distance = 3cm, minimum height=2em]
\tikzstyle{circ} = [draw, circle, node distance = 3cm, minimum height = 2em]

\begin{figure}[H]
\centering

\begin{tikzpicture}[node distance = 1cm, auto]
	%Place Nodes
	\node [block, text width = 1cm] (Kp) {$k_{p}$};
	\node [block, below of=Kp, node distance= 1.5cm, text width = 2.2cm] (integrator) {$k_{i} 	 \int_{{0}}^{t} e(\tau)d\tau$};
	\node [circ, left of= Kp, xshift = -1cm](sum){$\Sigma$};
	\node [block, below of=Kp, node distance = 4cm, text width = 2cm](Cavity){Cavity}; 
	\node [block, left of= sum, text width = 1cm] (set point){$\vec{E}_{\rm sp}$}; 
	\node [circ, right of= Kp] (sum1){$\Sigma$};

	%\path [line] (timeupdate) -- ++ (-4cm, 0) |- (forloop);
	\path [line] (sum) -- (-3cm,0) |- (integrator); 
	\path [line] (sum) -- node[near start, anchor = south]{$e(t)$}(Kp);
	%\path [line] (Kp) --; 
	%\path [line] (Kp) -- node[near end]{$\vec{K}_{g}$}(6cm,0); 
	\path [line] (integrator) -| (sum1);
	\path [line] (sum1) -- (4cm,0) |- (Cavity); 
	\path [line] (Cavity) -| node[near end]{$-\vec{E}_{\rm probe}$}(sum);
	\path [line] (set point) -- (sum); 
	\path [line] (Kp) -- (sum1);
	\path [line] (sum1) -- node{$\vec{K}_{g}$}(6cm,0);
	
	
\end{tikzpicture}
\caption{Flow chart of the PI controller}
\end{figure}

A set point $\vec{E}_{\rm sp}$ is chosen as a target prior to simulation. The error $e(t)$ is defined as the set point value minus the current $\vec{E}_{\rm probe}$ value:

\be
e(t) = \vec{E}_{\rm sp} - \vec{E}_{\rm probe}(t)
\ee

\noindent This error is passed onto a proportional and integral gain, with respective gain constants $k_{p}$ and $k_{i}$. Following the paths, we see that the incident wave amplitude $\vec{K}_{g}$ is altered according to: 
\be \label{eq: controller}
\vec{K}_{g} = k_{p} e(t) + k_{i} \int_{\tau_{0}}^{t} e(t) d\tau
\ee

\noindent This new $\vec{K}_{g}$ value is sent to the cavity.

\subsubsection{Software implementation}

\begin{figure}
\centering
\includegraphics[scale=0.25]{../figures/fpga_unit_test.png}
\caption{FPGA unit test.}
\label{fig:fpga_unit_test}
\end{figure}

In order to implement the PI controller in software, we need to discretize integral term of Equation ~\ref{eq: controller}. This term is shown below. 
\be
k_{i} \int_{\tau_{0}}^{t} e(\tau)d\tau
\ee

There are many ways to evaluate this integral. Most commonly utilized are right-hand, left-hand, and mid-point Riemann sums, Trapezoidal Rule, or Simpson's Rule. In our case, we will simply use a Trapezoidal Rule, which basically comes out to averaging the value of $e(t)$ at the previous and current time step. Considering a very small time step is used, this is a good approximation. Numerically, we have the following for the general case:
\be
\int_{\tau_{0}}^{\tau_{n}} f(\tau) d\tau \approx \Delta t
		    \left[\frac{f(\tau_{0})}{2} + f(\tau_{1}) + f(\tau_{2}) + ...
		      + f(\tau_{n-1}) + \frac{f(\tau_{n})}{2}\right]
\ee

\noindent where the subscript $n$ indicates the time step. To transform to actual time, one merely uses the relationship $\tau_{n} = n\Delta t$. Applied to our problem, we have the following: 

\be
k_{i} \int_{\tau_{0}}^{\tau_{n}} e(\tau)d\tau \approx k_{i}T
	      \left[\frac{e^{0}}{2} + e^{1} + e^{2} + ... 
		+ e^{n-1} + \frac{e^{n}}{2} \right]
\ee

\noindent where again the superscript notation indicates the time step. 

The overall update for $\vec{K}_{g}$ can then be expressed as:
\be \label{eq: controller_dis}
\vec{K}_{g}^{n+1} = k_{p} e^{n+1} + k_{i}T \sum_{k=0}^{n}\frac{e^{k+1}+e^k}{2}
\ee

Fig.~\ref{fig:fpga_unit_test} shows the result of a unit tests performed using the software implementation of eq.~\ref{eq: controller_dis}. The input signal and set-point are initially set to 0 and the FPGA routine is excercised individually, in absence of a plant or feedback loop. After 0.1 seconds of simulation, a step is introduced in the set-point (going from 0 to 1) and the drive signal is analyzed. Before any error signal is present in the integrator state, the drive signal is modulated by the set-point by a factor of $k_{\rm p}$. From that point the integrator behaviour can be observed, where the slope of the drive signal is given by $k_{\rm i}$. The settings and the measured values of the controller constants are indicated in Fig.~\ref{fig:fpga_unit_test}. 

\subsection{Amplifier}

\begin{figure}
\centering
\includegraphics[scale=0.35]{../figures/saturation_test.png}
\caption{Amplifier saturation test.}
\label{fig:saturation_test}
\end{figure}

Triode clipping is described with a harshness parameter c, such that the output signal v o from the tube based on its input vi varies according to

\begin{equation}
  \vec V_{\rm out} = \vec V_{\rm in} \cdot \left(1 + |\vec V_{\rm in}|^c\right)^{-1/c}
\end{equation}

The saturated output amplitude from this equation is 1. While some phase shift with drive level is observed, this effect is not yet included in the model. When c = 5, as provides a decent fit to measurements shown in figure 3, it takes about 1.5 (unitless) drive level from the control system to reach 97.5\% of this amplitude, 95\% of the power. This level should correspond to full scale digital output.

\subsection{Phase Shifter}

\begin{figure}
\centering
\includegraphics[scale=0.265]{../figures/phase_shift_test.png}
\caption{Phase shift test.}
\label{fig:phase_shfiter_test}
\end{figure}

Phase shifts are encountered at several stages of the model. A phase shifter module (not necessarily explicitly represented in block diagrams) is present in the software implemention and performs a phase shift of an input signal by an angle $\theta$ according to:

\begin{equation}
  \vec V_{\rm out} = \vec V_{\rm in} \cdot e^{j\theta}
\end{equation}

Fig.~\ref{fig:phase_shfiter_test} shows results of a unit tests performed on the phase shifter, where a sample signal is shifted by different angles and the phase differences are measured and compared with the phase angle parameter provided to the phase shifter software routine.

\subsection{LLRF Noise}

LLRF noise is dominated by ADC and preamplifier noise, which typically have broad-band (white) and 1/$f$ components. Here we only consider the broad-band component, as digital LLRF controllers (with {\it in-situ} calibration schemes) are effective at rejecting low frequency noise.

RF systems include several components such as the RF cavity, filters, digitizers, mixers (up and down converters), digital controller, etc. Some of these components are shown the the block diagram in Fig.~\ref{fig:Station_block_diagram}. Since different configurations of the RF system will result in differences in its frequency response, we prefer to express each noise sources in terms of its Power Spectral Density (PSD).

It is useful to express RF analog signal processing and ADC noise in dBc/Hz, where dBc is a logarithmic representation of the ratio between noise and carrier power. In the accelerator case, the carrier represents the nominal cavity signal, in turn something close to the full range of the \hbox{ADC}.  Normalization by the bandwidth gives a true performance number, independent of bandwidth (or equivalently, averaging). As mentioned earlier, we are only considering broad-band noise, so we will use a single noise value expressed in dBc/Hz, being constant over the entire frequency spectrum. This is a figure of merit for the RF measurement channel, which will vary depending on the amplifiers and ADCs used.
%Integrating this noise using the closed-loop frequency response one obtain the noise contribution in dBc (or Volts, or $\rm{V^2}$).

In our simulation models we use pseudo-random number generators to emulate noise sources, and typically express signals in normalized units, where the normalizing factor is the nominal cavity voltage. In the case of LLRF broadband noise, we choose normally distributed pseudo-random generated samples with zero mean and a variance calculated using the LLRF noise specs (in dBc/Hz), the full range of the ADC, and the operating bandwidth.

Let us use as an example the measured noise PSD of the LLRF4 system:
\begin{equation}
  \centering {\rm PSD}_{\rm LLRF} = -135\thinspace{\rm dBc/Hz} = 10^{-13.5} {\rm /Hz}
\end{equation}
The total normalized noise power in a bandwidth $B$ is then
%First we convert from dBc/Hz to dBc integrating over the operating bandwidth (B):
\begin{equation}
  \centering {\rm Noise}_{\rm LLRF} ={\rm PSD}_{\rm LLRF}\cdot B
  \label{eq:PSD_to_noise}
\end{equation}

\noindent If we use $1\thinspace\mu s$ simulation steps ($\Delta t_{sim}$):
\begin{equation}
  \centering B = \frac{1}{2} \cdot \frac{1}{\Delta t_{\rm sim}}=\frac{1}{2} \cdot 1\thinspace {\rm MHz} = 500\thinspace {\rm kHz}
\end{equation}

LLRF systems typically sample the cavity field faster than 1\thinspace MHz. Taking a more realistic sampling rate such as 100\thinspace MS/s ($f_S=100$\thinspace MHz), using $1\thinspace\mu s$ simulation steps would be equivalent to averaging 100\thinspace MHz samples by a factor $n=100$, which leads us to the same result:
\begin{equation}
  \centering B = \frac{1}{2} \cdot f_S \cdot \frac{1}{n} = \frac{1}{2} \cdot 100\thinspace{\rm MHz} \cdot \frac{1}{100} = 500\thinspace{\rm kHz}
\end{equation}

\noindent This means that we can use the simulation step to calculate the bandwidth, independent of the actual sampling rate.  While the choice of the latter has important practical implications, it doesn't affect the noise performance at this abstract level.

Once we know the bandwidth, we can obtain LLRF noise from the PSD using Eq.~\eqref{eq:PSD_to_noise}, which can be expressed as:
\begin{equation}
  \centering {\rm Noise}_{\rm LLRF} = \frac{P_{\rm Noise}}{P_{\rm ADC}}
  \label{eq:dBc_def}
\end{equation}
where $P_{\rm Noise}$ and $P_{\rm ADC}$ can be given in any self-consistent power units, such as $V^2$,
and $P_{\rm ADC}$ refers to the full-scale level of the \hbox{ADC}.
As mentioned earlier, we are interested in expressing the LLRF noise in normalized units, using the nominal cavity voltage as normalizing factor. Considering we design the RF system to provide the ADC with a dynamic range of 1.5 times the nominal cavity voltage as an example, 
%\begin{equation}
%  \centering V_{\rm Noise} (norm.) = \frac{V_{\rm Noise}(V)}{V_{\rm nom}(V)} = \frac{V_{\rm Noise}(V)}{V_{\rm ADC}(V)} \times \frac{V_{\rm ADC}(V)}{V_{\rm nom}(V)}=1.5 \times \sqrt{\frac{P_{\rm Noise}(V^2)}{P_{\rm ADC}(V^2)}}
%  \label{eq:v_noise_norm_def}
%\end{equation}
%where:\\
%$V_{\rm Noise} (norm.) = \mbox{LLRF noise voltage in normalized units,}$
%$V_{\rm nom}(V) = \mbox{Nominal cavity voltage in Volts} \quad \mbox{and} \quad $
%$V_{\rm ADC}(V) = \mbox{Full range of the ADC in Volts}$\\
%Combining Eqs.~\ref{eq:dBc_def} and~\ref{eq:PSD_to_noise}, we find:
%\begin{equation}
%  \centering \frac{P_{\rm Noise}(V^2)}{P_{\rm ADC}(V^2)} = 10^{\frac{Noise_{\rm LLRF}(dBc)}{10}} = 10^{\frac{PSD_{\rm LLRF}(dBc/Hz)}{10}} \times B
%  \label{eq:power_fraction}
%\end{equation}
%and substituting Eq.~\ref{eq:power_fraction} into~\ref{eq:v_noise_norm_def}, we get:
and defining $V_{\rm rms,norm}$ as the root-mean-square noise component added to
normalized cavity voltage,
\begin{equation}
  \centering V_{\rm rms,norm} = 1.5 \cdot \sqrt{{\rm PSD}_{\rm LLRF} \cdot B}
  \label{eq:v_noise_final}
\end{equation}

The above discussion is valid for baseband signals, but LLRF systems digitize the signal
around a carrier frequency, typically using $I/Q$ or near-$I/Q$ sampling of the measured cavity voltages. When sampled at $90^\circ$, the ADC produces a stream of $I, Q, -I, -Q$ samples, which gets repeated over and over again. We therefore get a stream of $I$ and $Q$ samples respectively at half the total rate. Eq.~\eqref{eq:v_noise_final} gives us the noise in one sample as a function of the bandwidth. In order to find the noise of the $I$ and $Q$ samples respectively we need to divide the total bandwidth by a factor of two. Considering $B$ in Eq.~\eqref{eq:v_noise_final} the total bandwidth, we get an identical noise level for both $I$ and $Q$ samples:
\begin{equation}
  I_{\rm rms,norm}= Q_{\rm rms,norm} = \frac{1}{\sqrt{2}}\cdot V_{\rm rms,norm}
  \label{eq:i_q_vs_v_rms}
\end{equation}
This relationship is also valid for near-$I/Q$ sampling after the DSP converts raw ADC samples to digital
$I$ and $Q$.

Now that we have the noise in the $I$ and $Q$ components of the measured cavity voltage, we can calculate the contribution of LLRF noise to both amplitude ($A_{\rm rms}$) and phase ($\Phi_{\rm rms}$) errors.
The general case is a nonlinear transformation that depends on the instantaneous value of the cavity field.
For small amounts of noise around an equilibrium setpoint, which is unity in our normalized treatment,
the results simplify to
\begin{equation}
  A_{\rm rms,norm} = I_{\rm rms,norm}
  \label{eq:a_noise}
\end{equation}
\begin{equation}
  \Phi_{\rm rms,rad} = Q_{\rm rms,norm}
  \label{eq:p_noise}
\end{equation}

Let us now calculate the expected noise levels in both amplitude and phase when using a LLRF4 system (${\rm PSD}_{\rm LLRF}=-135\thinspace$dBc/Hz) and a $1\thinspace\mu s$ simulation step ($B$=500\thinspace kHz). Combining Eqs.~\eqref{eq:v_noise_final} and~\ref{eq:i_q_vs_v_rms}, we get:

\begin{equation}
   I_{\rm rms,norm} = Q_{\rm rms,norm} =  1.5 \cdot \sqrt{\frac{1}{2} \cdot 10^{-13.5} \cdot 500 \times 10^3}= 1.334\times 10^{-4}
\end{equation}
which can then be approximated to amplitude and phase errors using Eqs.~\eqref{eq:a_noise} and~\eqref{eq:p_noise}, where the amplitude error is expressed in normalized units (normalizing factor being the nominal cavity voltage), and the phase error is expressed in radians.

\newpage

\section{Module}

\begin{figure}
\centering
\includegraphics[scale=0.3]{../figures/Cryomodule_block_diagram.png}
\caption{Module block diagram.}
\label{fig:Module_block_diagram}
\end{figure}

\subsection{Mechanical model}

The presence of an EM field inside the cavity generates forces on the cavity walls, resulting in deformation of the cavity and subsequently in a shift of the cavity resonant frequency~\cite{ref:delayen}, designated previously as detune frequency $\omega_{d_\mu}$. Each mode's fields generate a force proportional to $V_\mu^2 = |\vec V_\mu|^2$, and mechanical displacements influence each mode's instantaneous detune frequency.  Construct the previous section's $\omega_d$ as
a baseline $\omega_{d0}$ from the electrical mode solution ({\it e.g.}, $-2\pi (800\thinspace {\rm kHz})$ for the TTF cavity's $8\pi/9$ mode), plus a perturbation $\omega_{\mu}$ contributed from the mechanical mode deflections. 

Consider the electrical mode index $\mu$ to include not only electrical eigenmodes of one cavity, but modes of all cavities in the mechanical assembly ({\it e.g.}, cryomodule). Also include the dependence on piezoelectric actuator voltages $V_\kappa$. Then if the assembly's mechanical eigenmodes are indexed by $\nu$, mechanical forces $F_\nu$ and displacements $x_\nu$ of those eigenmodes are related to the electrical system by

\begin{equation}
F_\nu = \sum_\mu A_{\nu\mu} V_\mu^2 + \sum_\kappa B_{\nu\kappa}V_\kappa
\end{equation}

\begin{equation}
\omega_\mu = \sum_\nu C_{\mu\nu} x_\nu~~~,
\end{equation}

\noindent where $A$, $B$, and $C$ are constant matrices.

These expressions are understood to apply at every time instant; the quantities $V$, $F$, $x$, and $\omega$ all vary with time. The differential equation governing the dynamics of each mechanical eigenmode is that of a textbook second order low-pass filter.  In Laplace form,

\begin{equation}
k_\nu x_\nu = \frac{F_\nu}{ \displaystyle 1 + \frac{1}{Q_\nu}\frac{s}{\omega_\nu} + \left(\frac{s}{\omega_\nu}\right)^2}~~~,
\end{equation}

\noindent where $k_\nu$ is the spring constant. For computational purposes, we want it expressed in terms of the state-space formulation

\begin{equation}
\frac{d}{dt} \begin{pmatrix}
              x_{\nu} \\
              y_{\nu}
             \end{pmatrix}
             = \begin{pmatrix}
                a_{\nu} & -b_{\nu} \\
                b_{\nu} & a_{\nu} \\
               \end{pmatrix}
               \begin{pmatrix}
                x_{\nu}\\
                y_{\nu}
               \end{pmatrix}
               +c_{\nu} \begin{pmatrix}
                  0 \\
                  F_{\nu}
                 \end{pmatrix}
\label{eq: state_space}
\end{equation}

\noindent where a scaled velocity coordinate $y_{\nu}$ has been introduced. Convert the latter equation to Laplace form and solve to get

\begin{equation}
  \begin{pmatrix}
    x_{\nu} \\
    y_{\nu}
  \end{pmatrix}
  = 
  \begin{pmatrix}
    a_{\nu}-s& -b_{\nu} \\
    b_{\nu} & a_{\nu}-s \\
   \end{pmatrix}^{-1}
   \cdot
   \begin{pmatrix}
     0 \\
     F_{\nu}
  \end{pmatrix}
\end{equation}

\noindent Analytically invert that $2\times 2$ matrix, and multiply out to get

\begin{equation}
x_{\nu} = \frac{-b_{\nu}c_{\nu}F_{\nu}}{ (a_{\nu}-s)^2 + b_{\nu}^2}~~~.
\end{equation}

\noindent Equate coefficients with the earlier low-pass filter form, in the case $Q > \frac{1}{2}$, to get

\begin{equation}
a_{\nu}\pm jb_{\nu} = \omega_{\nu}\left( \frac{-1}{2Q_{\nu}} \pm j\sqrt{1-\frac{1}{4Q_{\nu}^2}}\right)
\end{equation}

\begin{equation}
c_{\nu} = -\frac{1}{k_{\nu}}\cdot\frac{a_{\nu}^2+b_{\nu}^2 }{ b_{\nu}} = - \frac{\omega_{\nu}^2}{k_{\nu} b_{\nu}}~~~.
\end{equation}

A deeper understanding of the forces and responses of a single electrical eigenmode $\mu$ of the cavity comes from Slater's perturbation theory.  For an eigenmode solution $\vec H_{\mu}(\vec r)\sin(\omega_{0_{\mu}} t)$, $\vec E_{\mu}(\vec r)\cos(\omega_{0_{\mu}} t)$ to Maxwell's equations in a closed conducting cavity (volume $V$), the stored energy $U_{\mu}$ is given by

\begin{equation} 
U_{\mu} = \int_V \left[ \frac{\mu_0}{4}H_{\mu}^2(\vec r)
                 +  \frac{\varepsilon_0}{4}E_{\mu}^2(\vec r) \right] dv~~~.
\end{equation}

Suppose a mechanical eigenmode $\nu$ involves small deflections $x_{\nu}\cdot \vec\xi(\vec r)$, where $x_{\nu}$ gives the amount of deflection, and the dimensionless quantity $\xi(\vec r)$ represents the mode shape. Both the force on the mode and the response to a deflection $x_{\nu}$ are given in terms of the Slater integral

\begin{equation}
 F_{\mu} = \int_S \left[ \frac{\mu_0    }{ 4}H^2(\vec r)
                 -  \frac{\varepsilon_0}{ 4}E^2(\vec r) \right]
   \vec n(\vec r) \cdot \vec\xi(\vec r) dS ~~~,
\end{equation}

\noindent where $\vec n(\vec r)$ is the normal vector to the cavity surface $S$, and $F_{\mu}$ directly gives the force. Note in particular the subtraction of $E$ and $H$ terms, contrasted with the addition in the energy integral.  Also notice the dot product of the deflection shape with the surface normal.  Then

\begin{equation}
\Delta\omega_{\mu} = -x\omega_{0_{\mu}} \left(\frac{F}{U}\right)_{\mu}
\end{equation}

\noindent and

\begin{equation}
F_{\mu} = \left(\frac{F}{U}\right)_{\mu} \cdot \frac{1}{(R/Q)_{\mu}\omega_{0_{\mu}}}  V_{\mu}^2~~~,
\end{equation}

\noindent where $\left(F/U\right)_{\mu}$ is a property of the electrical eigenmode, independent of amplitude, with units of~\hbox{m$^{-1}$}. Thus 

\begin{equation}
A_{\nu\mu} = \left(\frac{F}{U}\right)_{\mu} \cdot \frac{1}{(R/Q)_{\mu}\omega_{0_{\mu}}}~~~,
\end{equation}

\noindent and 

\begin{equation}
C_{\mu\nu} = -\omega_{0_{\mu}} \left(\frac{F}{U}\right)_{\mu}
\end{equation}

Slater's analysis above lets us express the static Lorentz response as

\begin{equation}
 \left(\frac{\Delta \omega}{ V^2}\right)_{\nu \mu} = \frac{C_{\mu\nu} A_{\nu\mu}}{ k_\nu} = - \left(\frac{F}{U}\right)_{\mu}^2 \cdot \frac{1}{k_\nu (R/Q)_{\mu}}
\end{equation}

\noindent correctly showing that this constant is always negative: the mode's static resonant frequency gets lower as it is filled. Summing over all mechanical modes $\nu$ gives the total DC response, often quoted in units of \hbox{Hz/(MV/m)$^2$}.

Using electrical measurements alone, it's not possible to constrain the scaling of $x_\nu$. It is therefore helpful to rescale $x_\nu$ and $F_\nu$ each by a factor of $\sqrt{k_\nu}$, and eliminate $k_\nu$ from the equations.  Instead of conventional units (m and N) for $x$ and $F$, they now both have units of $\sqrt{\rm Joules}$, so that $x\cdot F$ still represents energy. In this rescaled no-$k$ case,

\begin{equation}
A_{\nu\mu} = \frac{1}{ \omega_{0_{\mu}}} \sqrt{-\frac{1}{ (R/Q)_{\mu}}  \left(\frac{\Delta \omega}{ V^2}\right)_{\nu \mu}}
\end{equation}

\begin{equation}
 C_{\mu\nu} = -\omega_{0_{\mu}} \sqrt{-(R/Q)_{\mu}{ \left(\frac{\Delta \omega}{ V^2}\right)_{\nu \mu}}}\rlap{~~~.}
\end{equation}

It is perhaps an unexpected result that the cross-coupling between cavity modes ({\it e.g.}, excite the $\pi$ mode, measure $\Delta\omega$ for the $8\pi/9$ mode) is quantitatively predicted from measurements of each mode individually, with the exception of the choice of sign of the above radicals.  All that is required is confidence that mechanical modes are correctly identified and non-degenerate.

\subsubsection{Software implementation}

We now turn to the matrix Equation ~\ref{eq: state_space}. Two expressions appear:
\begin{equation}
  \frac{dx_{\nu}}{dt} = a_{\nu}x_{\nu} - b_{\nu}y_{\nu}
  \label{eq:dx/dt}
\end{equation}

\begin{equation}
  \frac{dy_{\nu}}{dt} = b_{\nu}x_{\nu} + a_{\nu}y_{\nu} + c_{\nu}F_{\nu}
  \label{eq:dy/dt}
\end{equation}

where: 

\be
a_{\nu} =  \frac{-\omega_{\nu}}{2Q_{\nu}} 
\ee
\be
b_{\nu} = \omega_{\nu} \sqrt{1 - \frac{1}{4Q_{\nu}^{2}}} 
\ee
\be
c_{\nu} = -\frac{\omega_{\nu}}{k_{\nu}b_{\nu}}
\ee

These displacement influence each mode's instantaneous  eigenmode frequency $\omega_{\mu}$ as following
\be
\label{eq_wmu}
\omega_\mu = \sum_\nu C_{\mu\nu} x_\nu
\ee

where $C$ is the coupling matrix from mechano to EM.

As in the case of the electrical eigenmodes, equations~\ref{eq:dx/dt} and~\ref{eq:dy/dt} are in the form of a 1st-order differential equation, for which the ODE integration is derived in Appendix~\ref{App:ODE_integration}, where Equations~\ref{eq:dx/dt} and~\ref{eq:dy/dt} can be written as Equation~\ref{eq:exp_diff2}, where:

\begin{equation}
 \vec V_{\rm out} = x_{\nu} \mbox{ , } p = a_{\nu} \mbox{ and, } \vec V_{\rm in} = -b_{\nu}y_{\nu}
\end{equation}

\noindent and,

\begin{equation}
 \vec V_{\rm out} = y_{\nu} \mbox{ , } p = a_{\nu} \mbox{ and, } \vec V_{\rm in} = b_{\nu}x_{\nu} + c_{\nu}F_{\nu}
\end{equation}

\noindent respectively.
% 
% 
% These electromagnetic fields interact mechanically.
% 
% The presence of an EM field inside the cavity generates forces on the cavity walls, resulting in deformation of the cavity and subsequently in a shift of the cavity resonant frequency~\cite{ref:delayen}, designated previously as detune frequency $\omega_{d_\mu}$.
% 
% Each mode's fields generate a force proportional to $V_\mu^2 = |\vec V_\mu|^2$ given by
% 
% \begin{equation}
% F_\nu = \sum_\mu A_{\nu\mu} V_\mu^2 
% \end{equation}
% 
% where $A$ is the coupling matrix from EM to mechano and the subscript $\nu$ indexes the mechanical eigenmodes.
% 
% This force is usually minimized using  piezoelectric actuators in the cavity. If we designate by $V_\kappa$ the amplitude of their voltages, the force becomes. 
% 
% \begin{equation}
% F_\nu = \sum_\mu A_{\nu\mu} V_\mu^2  + \sum_\kappa B_{\nu\kappa} V_\kappa
% \label{eq_F}
% \end{equation}
% 
% where $B$ is the coupling matrix from piezo to mechano.\\
% 
% The resulting mechanical displacements are governed by
% \begin{equation}
% \frac{d}{dt} \begin{pmatrix}
%               x_{\nu} \\
%               y_{\nu}
%              \end{pmatrix}
%              = \begin{pmatrix}
%                 a_{\nu} & -b_{\nu} \\
%                 b_{\nu} & a_{\nu} \\
%                \end{pmatrix}
%                \begin{pmatrix}
%                 x_{\nu}\\
%                 y_{\nu}
%                \end{pmatrix}
%                +c_{\nu} \begin{pmatrix}
%                   0 \\
%                   F_{\nu}
%                  \end{pmatrix}
% \label{eq: state_space}
% \end{equation}
% 
% where 
% 
% \begin{equation}
% a_{\nu} =  \frac{-\omega_{\nu}}{2Q_{\nu}} 
% \end{equation}
% \begin{equation}
% b_{\nu} = \omega_{\nu} \sqrt{1 - \frac{1}{4Q_{\nu}^{2}}} 
% \end{equation}
% 
% \begin{equation}
% c_{\nu} = -\frac{\omega_{\nu}}{k_{\nu}b_{\nu}}
% \end{equation}
% 
% These displacement influence each mode's instantaneous  eigenmode frequency $\omega_{\mu}$ as following
% 
% \begin{equation}
% \omega_\mu = \sum_\nu C_{\mu\nu} x_\nu
% \end{equation}
% 
% 
% where $C$ is the coupling matrix from mechano to EM.

\newpage

\section{Linac section}

\begin{figure}
\centering
\includegraphics[scale=0.45]{../figures/Linac_block_diagram.png}
\caption{Linac block diagram.}
\label{fig:Linac_block_diagram}
\end{figure}

\subsection{Particle tracking model}

\subsection{Beam-based feedback model}

\section{Conclusions}

\newpage

\begin{appendix}

\section{Appendix}

\subsection{ODE Integration of single-pole low-pass filter}
\label{App:ODE_integration}

\begin{figure}
\centering
\includegraphics[scale=0.265]{../figures/filter_step_response.png}
\caption{Comparison between numerical and analitycal filter step response.}
\label{fig:filter_step_response}
\end{figure}

% \begin{figure}
% \centering
% \includegraphics[scale=0.6]{../figures/filter_block_diagram.png}
% \caption{Generalized N-pole filter block diagram.}
% \label{fig:filter_block_diagram}
% \end{figure}

Start with the first order differential equation for a single-pole low pass filter. Its transfer function is expressed in Laplace form as

\begin{equation}
 TF(s)=\frac{\vec V_{\rm out}(s)}{\vec V_{\rm in}(s)} = \frac{1}{s-p}
\end{equation}

\noindent where $\vec V_{\rm in}$ and $\vec V_{\rm out}$ are the input and output signals, and $p$ is the pole location. Make the differentiation explicit, and rearrange to get a form consistent with state-variable numerical ODE integration,

\begin{equation}
 \frac{d\vec V_{\rm out}(t)}{dt} = \vec V_{\rm in}(t) + p\cdot \vec V_{\rm out}(t)
\label{eq:exp_diff}
\end{equation}

The simplest expression for a 'next' value at step $n$ in a discrete time \emph{ansatz} is

\begin{equation}
 \vec V_{\rm out}^n = \left( 1 + \Delta t \cdot p\right) \vec V_{\rm out}^{n-1} + \Delta t \cdot V_{\rm in}^n
\end{equation}

\noindent To improve convergence properties in the case where $\Delta t \cdot p$ is not tiny, approximate the trajectory of $\vec V_{\rm in}$ and $\vec V_{\rm out}$ as linear within a single time step. Specifically, assume that $\vec V_{\rm out}$ changes from $\vec V_{\rm out}^{n-1}$ to $\vec V_{\rm out}^n$, and $\vec V_{\rm in}$ changes from $\vec V_{\rm in}^{n-1}$ to $\vec V_{\rm in}^n$.

Using what is essentially the Trapezoidal Formula~\cite{ref:trapezoid}, our rendition of the discrete time approximation to the above differential equation becomes

\begin{equation} \label{eq:approx}
 	\vec V_{\rm out}^n=a \cdot \vec V_{\rm out}^{n-1}+\frac{1}{2}\cdot b \cdot (\vec V_{\rm in}^{n-1}+\vec V_{\rm in}^n)
\end{equation}

\begin{eqnarray}
	\nonumber \mbox{where} \quad a=\frac{1+ \frac{1}{2} \Delta t \cdot p}{1-\frac{1}{2}\Delta t \cdot p}, \quad \mbox{and} \quad b=\frac{\Delta t}{1-\frac{1}{2}\Delta t \cdot p}
\end{eqnarray}

\noindent $\Delta t$ is the simulation step duration, and $p$ is the pole location (a complex number). Cavity detuning is represented by a slight pole shifting into the imaginary direction.

In order to preserve scaling and have unity gain at DC, Eq.~\ref{eq:exp_diff} needs to be scaled by a factore of p, such that

\begin{equation}
 \frac{d\vec V_{\rm out}(t)}{dt} = p \cdot \vec V_{\rm in}(t) + p\cdot \vec V_{\rm out}(t)
\label{eq:exp_diff2}
\end{equation}

\noindent which is equivalent to scaling $\vec V_{\rm out}^n$ by a factor of $|p|$ in the software implementation.

This process is coded in C, and tested using a two-pole ow pass Butterworth filter. Given the transfer function $1/((s+1)^2+1)$, which has poles at (-1+1j) and (-1-1j), 
the step response is $1 - e^{-t} (\sin x + \cos x)$, for $t > 0$. This analytically known response is plotted in Fig.~\ref{fig:filter_step_response} along with the response obtained using the numerical model.


\end{appendix}

\newpage

\begin{thebibliography}{19}   % Use for  1-9  references

\bibitem{ref:model-paper}
M. Mellado Munoz, L. Doolittle, P. Emma, G. Huang, A. Ratti, C. Serrano, J. M. Byrd, ``A Dynamic feedback model for high repetition rate LINAC-Driver FELs,''
IPAC'12, New Orleans, LA, May 2012.

\bibitem{ref:litrack} 
P. Emma, K. Bane, L. Freitag, ``LiTrack : A Fast longitudinal phase space tracking code with graphical user interface'', PAC'05,  Knoxville, TN, May 2005.

\bibitem{ref:superfish}
Poisson Superfish Software, \url{http://laacg.lanl.gov/laacg/services/download_sf.phtml}

\bibitem{ref:montgomery}
C. G. Montgomery, R. H. Dicke. E. M Purcell, ``Principles of Microwave Circuits'', MIT Radiation Lab Series V8, 1947.

\bibitem{ref:svea}
Slowly varying envelope approximation (SVEA), \url{http://en.wikipedia.org/wiki/Slowly_varying_envelope_approximation}.

\bibitem{ref:schilcher}
T. Schilcher, ``Vector Sum Control of Pulsed Accelerating Fields in Lorentz Force Detuned Superconducting Cavities'', Hamburg 1998.

\bibitem{ref:cell_modes}
L. R. Doolittle, ``Understanding 5-cell mode structures'', JLab tech note CEBAF-TN-0120, May 1989.

\bibitem{ref:delayen}
J. R. Delayen, ``Ponderomotive Instabilities and Microphonics -- A Tutorial'', SRF'05, Ithaca, NY, July 2005.

\bibitem{ref:trapezoid}
Abramowitz and Stegun, ``Handbook of Mathematical Functions'', Formula 25.5.3, 1964.


\end{thebibliography}

\end{document}