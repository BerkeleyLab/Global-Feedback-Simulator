\documentclass[acus]{JAC2003}

%%
%%  This file was updated in April 2009 by J. Poole to be in line with Word tempaltes
%%
%%  Use \documentclass[boxit]{JAC2003}
%%  to draw a frame with the correct margins on the output.
%%
%%  Use \documentclass{JAC2003}
%%  for A4 paper layout
%%

\usepackage{graphicx}
\usepackage{booktabs}

%%
%%   VARIABLE HEIGHT FOR THE TITLE BOX (default 35mm)
%%

\setlength{\titleblockheight}{27mm}

\begin{document}
\title{NEW GENERATION OF LLRF AND BEAM-BASED FEEDBACK STABILITY MODELS}

\author{A. F. Queiruga, D. S. Driver, University of California, Berkeley, CA 94720, USA\\
Q. Llimona, Universitat Pompeu Fabra, Barcelona, 08018, Spain\\
L. R. Doolittle, C. Serrano\thanks{CSerrano@lbl.gov}, LBNL, Berkeley, CA 94720, USA}

\maketitle

\begin{abstract}
    All RF groups in the accelerator community use modeling to some extent for LLRF development, and LBNL has
traditionally invested a significant effort in this area for a number of reasons. It is possible to solve many
issues in the control algorithms in a comfortable software environment, often well before the availability of production
systems. Software and hardware designs can be combined in simulation, easing the transition to final hardware system
integration.  Over time we have built a solid base of generic models which can be applied to a variety of machines and
configuration schemes. The low-level implementation of the models has been re-implemented and extended to include
elements in beam-based feedback loops and optimized to improve computation speed, memory usage, scalability and
flexibility. The way they are configured and interact with high-level software has has been greatly improved to provide clean APIs
and a convenient configuration scheme accessible from user-friendly tools such as GUIs and websites.
\end{abstract}

\section{INTRODUCTION}


\section{CONCLUSIONS}

\begin{thebibliography}{9}   % Use for  1-9  references
%\begin{thebibliography}{99} % Use for 10-99 references

\bibitem{original-model-ref}
M. Mellado Munoz, L. Doolittle, P. Emma, G. Huang, A. Ratti, C. Serrano, J. M. Byrd, ``A Dynamic feedback model for high repetition rate LINAC-Driver FELs,''
IPAC'12, New Orleans LA, May 2012.


\end{thebibliography}

\end{document}
